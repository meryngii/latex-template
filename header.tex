%%%%% header.tex %%%%%

% 古いパッケージを警告
\RequirePackage[l2tabu, orthodox]{nag}

\documentclass[10pt,a4j]{jsarticle}

% パッケージの読み込み %{{{

% graphicx, color : 画像に必要
\usepackage[dvipdfmx]{graphicx}
\usepackage[dvipdfmx]{color}

% amsmath : 数式に必要
\usepackage{amsmath}

% fontenc: アンダーバーを検索可能にする
\usepackage[T1]{fontenc}
% Computer Modernを使用
\usepackage{lmodern}
% Times系フォントを使用
\IfFileExists{newtxtext.sty}{
\usepackage{newtxtext,newtxmath}
}{
\usepackage{txfonts}
}

% SI単位系を表示
\usepackage{siunitx}

% 用紙余白を指定
\usepackage[top=20truemm,bottom=25truemm,left=20truemm,right=20truemm]{geometry}
% 段組み
\usepackage{multicol}
% 複数行コメント
\usepackage{comment}
% モジュール読み込み
\usepackage{import}
% arrayとtabular環境を改善
\usepackage{array}
% 枠線をつける
\usepackage{ascmac}
\usepackage{fancybox}
%% 数式環境で太字を使う (\bmmax, \hmmaxの設定に注意)
%\usepackage{bm}
% 図表位置を[H]で強制指定
\usepackage{here}
% float外で図表番号を表示する (captionof)
\usepackage{capt-of}
% ソースコードを整形して表示
\usepackage{listings}
% ページをまたいだ要素でも左右に余白を付けられる (adjustwidth)
\usepackage{changepage}
% URLを表示
\usepackage{url}
% 複数の図表をまとめて表示
\usepackage{subfigure}

%% 表の行に色をつける (rowcolor)
%\usepackage{colortbl}
%% 参考文献番号を上に表示する
%\usepackage{overcite}

%}}}

% 各種数学記号 %{{{
% \diff : 微分 d
\newcommand{\diff}{\mathrm{d}}
% \divergence, \grad, \rot
\newcommand{\divergence}{\mathrm{div}\,}
\newcommand{\grad}{\mathrm{grad}\,}
\newcommand{\rot}{\mathrm{rot}\,}

% \laplace, \fourier
\newcommand{\laplace}{\mathcal{L}}
\newcommand{\fourier}{\mathcal{F}}

% \Celsius
\newcommand{\Celsius}{{}^\circ\mathrm{C}}


%}}}

% 相互参照コマンド %{{{
\newcommand{\figref}[1]{図~\ref{fig:#1}}
\newcommand{\eqnref}[1]{式 (\ref{eqn:#1})}
\newcommand{\tabref}[1]{表~\ref{tab:#1}}
\newcommand{\secref}[1]{\S\,\ref{sec:#1}}
%}}}

% 表番号の形式変更 %{{{
\makeatletter
\renewcommand{\thetable}{\thesection.\arabic{table}}
\@addtoreset{table}{section}
\makeatother
%}}}

% 図番号の形式変更 %{{{
\makeatletter
\renewcommand{\thefigure}{\thesection.\arabic{figure}}
\@addtoreset{figure}{section}
\makeatother
%}}}

% 数式番号の形式変更 %{{{
\makeatletter
\renewcommand{\theequation}{\thesection.\arabic{equation}}
\@addtoreset{equation}{section}
\makeatother
%}}}

% 参考文献の形式 %{{{
\bibliographystyle{tipsj}
%}}}

% citeの設定 %{{{
\makeatletter
\def\@cite#1{[#1]}
\makeatother
%}}}

% footnoteをダガーで表示 %{{{
\renewcommand{\thefootnote}{\fnsymbol{footnote}}
%}}}

% listingの設定(lstlisting) %{{{
\lstset{
    breaklines=true, % 行が長くなったら改行する
    basicstyle=\ttfamily\footnotesize,
    commentstyle={\itshape \color[cmyk]{1,0.4,1,0}},
    classoffset=1,
    keywordstyle={\bfseries \color[cmyk]{0,1,0,0}},
    stringstyle={\ttfamily \color[rgb]{0,0,1}},
    frame=tRBl,
    framesep=5pt,
    showstringspaces=false,
    numbers=left,
    stepnumber=1,
    numberstyle=\tiny,
    tabsize=2,
    xrightmargin=0zw,
    xleftmargin=3zw
}
%}}}

