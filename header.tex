%%%%% header.tex %%%%%

\documentclass[10pt,a4j]{jsarticle}

% パッケージの読み込み %{{{
\usepackage[dvipdfmx]{graphicx}
\usepackage{color}
\usepackage{amsmath,amssymb}
% 用紙余白を指定
\usepackage[top=25truemm,bottom=30truemm,left=25truemm,right=25truemm]{geometry}
% 段組み
\usepackage{multicol}
% 複数行コメント
\usepackage{comment}
% モジュール読み込み
\usepackage{import}
% arrayとtabular環境を改善
\usepackage{array}
% 枠線をつける
\usepackage{ascmac}
\usepackage{fancybox}
% 数式環境で太字を使う
\usepackage{bm}
% 図表位置を[H]で強制指定
\usepackage{here}
%}}}

% 各種数学記号 %{{{
% \diff : 微分 d
\newcommand{\diff}{\mathrm{d}}
% \divergence, \grad, \rot
\newcommand{\divergence}{\mathrm{div}\,}
\newcommand{\grad}{\mathrm{grad}\,}
\newcommand{\rot}{\mathrm{rot}\,}

% \laplace, \fourier
\newcommand{\laplace}{\mathcal{L}}
\newcommand{\fourier}{\mathcal{F}}

% \Celsius
\newcommand{\Celsius}{{}^\circ\mathrm{C}}

%}}}

% 図表番号の参照 %{{{
\newcommand{\reffigure}[1]{図~\ref{figure:#1}}
\newcommand{\refequation}[1]{式 (\ref{equation:#1})}
\newcommand{\reftable}[1]{表~\ref{table:#1}}
%}}}

% 数式番号の形式変更 %{{{
\makeatletter
\renewcommand{\theequation}}}

% 参考文献の形式 %{{{
\bibliographystyle{tipsj}
%}}}

